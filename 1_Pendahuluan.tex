\chapter{\uppercase{Pendahuluan}}\label{pendahuluan}
\doublespacing
\section{Latar Belakang}
Dokumen ini adalah template Proposal Penelitian Tugas Akhir Jurusan Teknofisika Nuklir STTN-Batan. Bagian Pendahuluan berisi Latar Belakang, Rumusan Masalah, Batasan Penelitian, Keaslian Penelitian, Tujuan Penelitian dan Manfaat Penelitian. Uraikan artikel-artikel yang menjadi latar belakang penelitian ini dilakukan. Tuliskan apa yang sudah dilakukan oleh peneliti lain. Referensi menggunakan cara HAVARD. Kutipan diurutkan berdasar urut abfabet.  Tidak disarankan ada gambar di bagian pendahuluan. Jika sangat diperlukan adanya gambar, maka harus sesuai ketentuan. Lihat di bab selanjutnya mengenai gambar.

\section{Rumusan Masalah}
Perumusan masalah menjelaskan masalah yang ada sehingga penelitian perlu dilakukan. Diuraikan berdasarkan latar belakang penelitian yang sudah dijelaskan di sub sebelummnya. Perumusan masalah hanya berupa masalah, bukan apa yang dilakukan. Perumusan masalah biasanya bernada negatif yang perlu diselesaikan pada tujuan penelitian. Tidak ada sitasi. Bagian ini murni tulisan sendiri.

\section{Batasan Masalah}
Jelaskan apa yang akan dilakukan dan apa yang tidak akan dilakukan dalam penelitian. Batasan juga dapat menjelaskan batasan alat, bahan, ataupun data penelitian.

Pembatasan masalah dalam penelitian ini perlu dilakukan agar dalam pembahasannya lebih terarah dan sistematis.

\section{Keaslian Penelitian}
Jelaskan novelty atau kontribusi penelitian di sini. Uraikan dari latar belakang dengan lebih menonjolkan peran penelitian anda. Akan lebih banyak sitasi di sini dibandingkan di bagian latar belakang. Jika ada kekurangan peneliti lain, sehingga proposal penelitian ini penting, ungkapkan di bagian ini. Cara penulisan keaslian penelitian boleh dalam bentuk tabel, \textit{fish bone diagram} maupun narasi. Penulisan Nama-Tabel menggunakan tabular seperti contoh dibawah, sedangkan untuk mengacu nama tabel gunakan reference pada label seperti ini (Tabel \ref{tabel Al}). %\ref{nama_tabel}.

\begin{table}[!htb] \index{Hasil HVL}
	\begin{center}
		\caption[Contoh HVL]{Hasil Percobaan HVL Alumunium vs $^{60}$Co}
		\label{tabel Al}
		\begin{tabular}{ccccc}
			\hline
			Ketebalan (inch) & Cacahan rata-rata & N (cps) & N\textsubscript{0} (cps) & $\ln\left(\frac{N}{N_0}\right)$\\
			\hline
			0.125 & 358.67 & 11.15 & 15.02 & -0.2934\\
			0.08 & 356 & 11.06 & 15.02 & -0.3024\\
			0.02 & 413.33 & 12.98 & 15.02 & -0.1444\\
			\hline
		\end{tabular}
	\end{center}
\end{table}

\section{Tujuan Penelitian} \label{tujuan}
Judul bab atau subbab baru tidak boleh sendiri di bawah. Jika terpaksa sendirian di bawah, enter satu kali sehingga masuk halaman baru. 

Tujuan penelitian menerangkan apa yang ingin dicapai dari penelitian. Bagian ini adalah jawaban dari perumusan masalah yang diuraikan sebelumnya. Jika dimungkinkan, jelaskan ukuran keberhasilan dari penelitian yang akan dilakukan. Tujuan penelitian dapat lebih dari satu, namun tetap menjawab/menyelesaikan masalah. Jika tujuan penelitian lebih dari satu, tuliskan dengan bullet atau numbering.

\section{Manfaat Penelitian}
Jelaskan manfaat yang diperoleh jika penelitian berhasil. Jelaskan baik dari sisi ilmu pengetahuan maupun kemanfaatannya bagi masyarakat. Akan sangat baik jika ada manfaat bagi bangsa dan negara.