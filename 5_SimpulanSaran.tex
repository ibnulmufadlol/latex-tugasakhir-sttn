	\chapter{\uppercase{kesimpulan dan saran}} \label{bab5}
\section{Kesimpulan} \label{kesimpulan}
Pada sub-bab ini dituliskan kesimpulan hasil penelitian atau kesimpulan TA. Kesimpulan harus ditulis berdasarkan hasil penelitian, pembahasan, dan temuan yang telah ditulis pada bab sebelumnya yang tentu saja disesuaikan dengan tujuan penelitian atau TA. Jangan menyimpulkan sesuatu yang tidak ada di dalam pembahasan yang telah dibuat. Kesimpulan dibuat dengan singkat dan jelas dengan urutan yang sebisa mungkin sesuai dengan tujuan penelitian (tertulis pada sub-bab tujuan penelitian).

Kesimpulan merupakan intisari dari pembahasan yang bersifat lebih general. Kesimpulan harus disesuaikan dengan hipotesis dan atau tujuan penelitian, dan juga harus dibuktikan di Bab IV (Hasil dan Pembahasan).

Kesimpulan boleh diberi nomor atau boleh juga tidak menggunakan nomor. Jika menggunakan nomor maka sesuai dengan template ini.

Jika tidak menggunakan nomor maka gunakan alenia-alenia sebagaimana ketentuan paragraph penulisan tesis seperti bab-bab sebelumnya.

\section{Saran} \label{saran}
Pada sub-bab ini dituliskan saran yang diusulkan oleh penulis. Dalam hal ini ada dua jenis saran:
\begin{enumerate}
	\item Saran untuk penelitian selanjutnya / kajian lanjutan. Saran jenis ini diberikan pada TA yang bersifat penelitian dan modelling. Saran ini berisi berbagai hal yang belum dilakukan, atau belum selesai dilakukan, atau berbagai hal yang merupakan lanjutan penelitian yang telah dilakukan dalam TA ini. Saran yang dibuat harus berdasarkan pembahasan serta kesimpulan yang telah dibuat. Jangan menyarankan sesuatu yang berada di luar jangkauan pembahasan dan kesimpulan yang dibuat.
	\item Saran terhadap perbaikan sistem yang dibahas dalam TA / practical implication. Saran jenis ini diberikan pada TA yang bersifat studi kasus. Saran ini berisi berbagai hal yang harus dilakukan untuk perbaikan sistem yang telah dibahas dalam sub-bab pembahasan dan kesimpulan. Saran yang diberikan hasus masuk akal dan mungkin untuk dilakukan / diaplikasikan. Saran ini tentunya berdasarkan temuan yang diperoleh dalam pembahasan dan disimpul-kan dalam sub-bab kesimpulan. Jangan memberikan saran yang berbeda / menyimpang dengan apa yang dibahas dan disimpulkan pada sub-bab pembahasan dan kesimpulan.
\end{enumerate}