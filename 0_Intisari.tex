\chapter*{INTI SARI}
\fontsize{14pt}{1.5em}\selectfont
\begin{center}
	\uppercase{\textbf{judul tugas akhir}}\\
\end{center}
\begin{tabular}{lcl}
	Nama & : & .................................................\\
	NIM & : & .................................................\\
	Pembimbing I &	: & .................................................\\
	Pembimbing II &	: & .................................................\\
\end{tabular}

\singlespacing
\noindent Disertasi ini membahas manifold \emph{upper half-plane} empat dimensi dan aplikasinya pada supergravitasi $N=2$ di lima dimensi. Manifold ini merupakan generalisasi dari metrik Joyce, yang merupakan salah satu referensi utama di riset ini. Solusi Pembahasan manifold \emph{upper half-plane} dimulai dengan analisis kestabilan dari potensial dan Hamiltonian. Hasil dari analisis tersebut menunjukkan bahwa potensial berkaitan dengan Hamiltonian saat potensialnya berada di titik kritis. Analisis dari kasus khusus metrik \emph{upper half-plane} untuk satu parameter menunjukkan bahwa metriknya tidak mungkin memenuhi syarat Einstein. Persamaan gerak dari aksi Riemann-Hilbert-nya dapat dilinearisasi sehingga memberikan solusi untuk aproksimasi orde pertama. Pembahasan di akhir disertasi mengarah kepada aplikasi metrik Joyce ke solusi \emph{domain wall} pada supergravitasi $N=2$ di lima dimensi. Dalam aplikasinya, solusi \emph{domain wall} akan mereduksi permasalahan empat dimensi menjadi permasalahan dua dimensi di ruang \emph{upper half-plane}. Syarat kestabilan dari persamaan aliran diperoleh dari matriks Hessian.

\noindent\textbf{Kata kunci:} \emph{domain wall}, Joyce, stabilitas, \emph{upper half-plane}